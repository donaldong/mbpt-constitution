\documentclass[20pt]{constitution}
\usepackage{mathpazo}
\begin{document}
\title{Monterey Bay Programming Team Constitution (*Draft)}
\date{May 2018}
\maketitle
\setcounter{tocdepth}{0}
\tableofcontents
\newpage

\article{Name}
The name of the organization is Monterey Bay Programming Team ({\bf MBPT}).

\article{Purpose}
\section{}
MBPT is to bring together people who are interested in solving coding puzzles,
people who are motivated to become a stronger coder, 
and people who enjoyed coding in a competitive environment, 
across all majors and backgrounds.
\section{}
MBPT is to engage and motivate its members to come up ideas and solutions to 
different types of problems, and explore classic data structures and algorithms.
\section{}
MBPT is to prepare its members for upper-division computer science classes, 
technical job interviews, and programming contests such as ACM-ICPC.

\article{Membership}
MBPT has 4 classes of membership: Guest, Regular Members, Active Members, and Core Members.
The requirements for each class are defined as follows:
\section{Guest}
\begin{itemize}
    \item Basic knowledge of at least one programming language.
\end{itemize}
\section{Regular Members}
\begin{enumerate}
    \item Being familiar with at least one programming language.
    \item Attending at least one meeting per week.
\end{enumerate}
\section{Active Members}
\begin{enumerate} 
    \item Being familiar with at least one programming language.
    \item Attending at least {\bf two} meetings per week.
    \item Being active during the meetings.
\end{enumerate}
\section{Core Members}
\begin{enumerate} 
    \item Being fluent with at least one programming language.
    \item Attending at least {\bf two} meetings per week.
    \item Being active during the meetings.
    \item Finishing weekly assigned coding objective.
\end{enumerate}

\article{Leadership}
The elected officers shall be: President, Vice-President, Treasurer, and Secretary.
In general, officers should be a {\bf core member} but do not have voting privileges during the election. 
Officers should be active on {\bf \#progteam} channel.
The position specified duties and powers of the officers shall be as follows:

\section{President}
\begin{enumerate}
    \item President should consult with the officers and make the final decisions.
    \item President should ensure MBPT runs smoothly, in the right direction. 
\end{enumerate}

\section{Vice-President}
\begin{enumerate}
    \item Vice-President should be planning the events and meeting materials with the President.
\end{enumerate}

\section{Treasurer}
\begin{enumerate}
    \item Treasurer should be organizing resources (developers, designers, and sponsors) 
    to promote MBPT, such as the MBPT web project.
\end{enumerate}

\section{Secretary}
\begin{enumerate}
    \item Secretary should be organizing public relations, such as ICC, other departments, and nearby colleges. 
\end{enumerate}

\article{Meetings}
\section{}
Regular meetings shall be scheduled bi-weekly during the academic year.
\section{}
Meeting time will be determined based off the schedules of the majority of the active members and core members.
Members with such privileges must inform their schedule preferences 2 weeks before the new semester begins.
\section{}
Additional meetings will be added if there are enough demand. 

\article{Advisors}
\section{}
The MBPT shall appoint a faculty or staff member with Computer Science background at CSUMB,
to serve as the university advisor to MBPT. 

\article{Voting}
\section{}
Guests do not have voting privileges.
Regular members, active members and core members have voting privileges.
\section{}
The votes from regular members count as half of the vote.
The votes from active members and core members count as a full vote. 
\section{}
Election will be hold after the week of ACM-ICPC.

\article{Amending and Ratify the Constitution}
\section{}
Proposed amendments to these bylaws shall be presented to the membership, 
in writing, two meetings prior to the meeting where the amendment will be voted upon.  
\section{}
Amending and ratify the constitution could happen at any time. It must be approved by all the leadership members,
and more than 2/3 of the active members and core members.
\section{}
All the regular members, active members, and core members have to resign the updated constitution
to remain the membership status.

\end{document}
